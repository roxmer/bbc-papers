\section{BiobankCLoud Regulatory Framework}
“Regulatory and Ethical Requirements for Biobanking Data Storage and Analysis”
Karolinska Institute (KI) has created the ethical and regulatory framework for a controlled and regulated data management and sharing that will guarantee that the BiobankCloud is ethically viable and can be integrated with European biobanks. In order to uphold the highest standard for privacy within the BiobankCloud, a diverse set of tools need to be implemented; legal tools, ethical tools and technical tools. All three categories stem from the EU Data Protection Directive (95/46/EC, the Directive), which sets out a harmonized common standards for the entire EU. Within the BiobankCloud-project as whole, technical solutions to safeguard privacy is central. Our main focus has been on the legal and technical tools. 
The legal tools implemented in the platform focus on allocating the responsibilities stemming from the Data Protection Directive between the involved entities in a clear and concise manner.  The Directive stipulates that if the data owner (the controller) does not conduct the processing of the personal data him/herself but leaves this to a service provider such as the BiobankCloud (the processor), the processing of data must be governed by a contract or legal act binding the data owner to the service-provider. A transparent and reliable chain of command is thus set out in a draft agreement.  As further general safeguard, no data will be processed that can be linked to any individual person.
The main ethical tool developed within the platform is a framework expressed in a Model Data Management Policy (MDMP). The model identifies management elements, formal rules and processes to be implemented in the platform in order to guarantee a consistent ascertainment of all ethical and regulatory requirements that should be fulfilled by the platform itself and by the users of the BiobankCloud platform. This policy was translated into a data model that facilitates its implementation in any informatics system. 
BBMRI-ERIC has created the Minimum Information About Biobank data Sharing (MIABIS) [1] which is the de-facto standard to represent biobank data in the BBMRI-ERIC community. This standard has evolved to MIABIS 2.0 and has been implemented in the platform as part of the data model for representing biobank data (BiobankCloud LIMS). 
Both the legal and ethical tools will be facilitated by the adoption of the BMRI-ERIC of the BiobankCloud platform as part of its IT Common Services. It will guarantee a continued maintenance of the platform beyond the project lifetime and at the same time it will provide a trusted environment for the platform contributing in this way, to spreading the use of the BiobankCloud platform in other bio-medical research and biobank networks. BBMRI-ERIC will also act as the bridge between the BiobankCloud platform and industrial partners through BBMRI-ERIC Expert Centres. 

